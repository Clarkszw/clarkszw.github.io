% Options for packages loaded elsewhere
\PassOptionsToPackage{unicode}{hyperref}
\PassOptionsToPackage{hyphens}{url}
%
\documentclass[
]{article}
\usepackage{amsmath,amssymb}
\usepackage{lmodern}
\usepackage{iftex}
\ifPDFTeX
  \usepackage[T1]{fontenc}
  \usepackage[utf8]{inputenc}
  \usepackage{textcomp} % provide euro and other symbols
\else % if luatex or xetex
  \usepackage{unicode-math}
  \defaultfontfeatures{Scale=MatchLowercase}
  \defaultfontfeatures[\rmfamily]{Ligatures=TeX,Scale=1}
\fi
% Use upquote if available, for straight quotes in verbatim environments
\IfFileExists{upquote.sty}{\usepackage{upquote}}{}
\IfFileExists{microtype.sty}{% use microtype if available
  \usepackage[]{microtype}
  \UseMicrotypeSet[protrusion]{basicmath} % disable protrusion for tt fonts
}{}
\makeatletter
\@ifundefined{KOMAClassName}{% if non-KOMA class
  \IfFileExists{parskip.sty}{%
    \usepackage{parskip}
  }{% else
    \setlength{\parindent}{0pt}
    \setlength{\parskip}{6pt plus 2pt minus 1pt}}
}{% if KOMA class
  \KOMAoptions{parskip=half}}
\makeatother
\usepackage{xcolor}
\usepackage[margin=1in]{geometry}
\usepackage{color}
\usepackage{fancyvrb}
\newcommand{\VerbBar}{|}
\newcommand{\VERB}{\Verb[commandchars=\\\{\}]}
\DefineVerbatimEnvironment{Highlighting}{Verbatim}{commandchars=\\\{\}}
% Add ',fontsize=\small' for more characters per line
\usepackage{framed}
\definecolor{shadecolor}{RGB}{248,248,248}
\newenvironment{Shaded}{\begin{snugshade}}{\end{snugshade}}
\newcommand{\AlertTok}[1]{\textcolor[rgb]{0.94,0.16,0.16}{#1}}
\newcommand{\AnnotationTok}[1]{\textcolor[rgb]{0.56,0.35,0.01}{\textbf{\textit{#1}}}}
\newcommand{\AttributeTok}[1]{\textcolor[rgb]{0.77,0.63,0.00}{#1}}
\newcommand{\BaseNTok}[1]{\textcolor[rgb]{0.00,0.00,0.81}{#1}}
\newcommand{\BuiltInTok}[1]{#1}
\newcommand{\CharTok}[1]{\textcolor[rgb]{0.31,0.60,0.02}{#1}}
\newcommand{\CommentTok}[1]{\textcolor[rgb]{0.56,0.35,0.01}{\textit{#1}}}
\newcommand{\CommentVarTok}[1]{\textcolor[rgb]{0.56,0.35,0.01}{\textbf{\textit{#1}}}}
\newcommand{\ConstantTok}[1]{\textcolor[rgb]{0.00,0.00,0.00}{#1}}
\newcommand{\ControlFlowTok}[1]{\textcolor[rgb]{0.13,0.29,0.53}{\textbf{#1}}}
\newcommand{\DataTypeTok}[1]{\textcolor[rgb]{0.13,0.29,0.53}{#1}}
\newcommand{\DecValTok}[1]{\textcolor[rgb]{0.00,0.00,0.81}{#1}}
\newcommand{\DocumentationTok}[1]{\textcolor[rgb]{0.56,0.35,0.01}{\textbf{\textit{#1}}}}
\newcommand{\ErrorTok}[1]{\textcolor[rgb]{0.64,0.00,0.00}{\textbf{#1}}}
\newcommand{\ExtensionTok}[1]{#1}
\newcommand{\FloatTok}[1]{\textcolor[rgb]{0.00,0.00,0.81}{#1}}
\newcommand{\FunctionTok}[1]{\textcolor[rgb]{0.00,0.00,0.00}{#1}}
\newcommand{\ImportTok}[1]{#1}
\newcommand{\InformationTok}[1]{\textcolor[rgb]{0.56,0.35,0.01}{\textbf{\textit{#1}}}}
\newcommand{\KeywordTok}[1]{\textcolor[rgb]{0.13,0.29,0.53}{\textbf{#1}}}
\newcommand{\NormalTok}[1]{#1}
\newcommand{\OperatorTok}[1]{\textcolor[rgb]{0.81,0.36,0.00}{\textbf{#1}}}
\newcommand{\OtherTok}[1]{\textcolor[rgb]{0.56,0.35,0.01}{#1}}
\newcommand{\PreprocessorTok}[1]{\textcolor[rgb]{0.56,0.35,0.01}{\textit{#1}}}
\newcommand{\RegionMarkerTok}[1]{#1}
\newcommand{\SpecialCharTok}[1]{\textcolor[rgb]{0.00,0.00,0.00}{#1}}
\newcommand{\SpecialStringTok}[1]{\textcolor[rgb]{0.31,0.60,0.02}{#1}}
\newcommand{\StringTok}[1]{\textcolor[rgb]{0.31,0.60,0.02}{#1}}
\newcommand{\VariableTok}[1]{\textcolor[rgb]{0.00,0.00,0.00}{#1}}
\newcommand{\VerbatimStringTok}[1]{\textcolor[rgb]{0.31,0.60,0.02}{#1}}
\newcommand{\WarningTok}[1]{\textcolor[rgb]{0.56,0.35,0.01}{\textbf{\textit{#1}}}}
\usepackage{graphicx}
\makeatletter
\def\maxwidth{\ifdim\Gin@nat@width>\linewidth\linewidth\else\Gin@nat@width\fi}
\def\maxheight{\ifdim\Gin@nat@height>\textheight\textheight\else\Gin@nat@height\fi}
\makeatother
% Scale images if necessary, so that they will not overflow the page
% margins by default, and it is still possible to overwrite the defaults
% using explicit options in \includegraphics[width, height, ...]{}
\setkeys{Gin}{width=\maxwidth,height=\maxheight,keepaspectratio}
% Set default figure placement to htbp
\makeatletter
\def\fps@figure{htbp}
\makeatother
\setlength{\emergencystretch}{3em} % prevent overfull lines
\providecommand{\tightlist}{%
  \setlength{\itemsep}{0pt}\setlength{\parskip}{0pt}}
\setcounter{secnumdepth}{-\maxdimen} % remove section numbering
\ifLuaTeX
  \usepackage{selnolig}  % disable illegal ligatures
\fi
\IfFileExists{bookmark.sty}{\usepackage{bookmark}}{\usepackage{hyperref}}
\IfFileExists{xurl.sty}{\usepackage{xurl}}{} % add URL line breaks if available
\urlstyle{same} % disable monospaced font for URLs
\hypersetup{
  pdftitle={Data Analysis Project},
  pdfauthor={Zhiwen Shi},
  hidelinks,
  pdfcreator={LaTeX via pandoc}}

\title{Data Analysis Project}
\author{Zhiwen Shi}
\date{2022/10/21}

\begin{document}
\maketitle

\hypertarget{description}{%
\subsubsection{Description}\label{description}}

This is one of the project reports in my portfolio about data analysis
using R.

The aim for the project is to find the best market to invest the
advertisement for the e-leaning products.

The skills are covering data extraction, data cleaning, data
visualization, data analysis and strategy proposal.

\hypertarget{introduction}{%
\section{Introduction}\label{introduction}}

\hypertarget{finding-the-two-best-markets-to-advertise-in-an-e-learning-product}{%
\subsubsection{Finding the Two Best Markets to Advertise in an
E-learning
Product}\label{finding-the-two-best-markets-to-advertise-in-an-e-learning-product}}

In this project, we'll aim to find the two best markets to advertise our
product in --- we're working for an e-learning company that offers
courses on programming. Most of our courses are on web and mobile
development, but we also cover many other domains, like data science,
game development, etc.

\hypertarget{understanding-the-data}{%
\section{Understanding the Data}\label{understanding-the-data}}

To avoid spending money on organizing a survey, we'll first try to make
use of existing data to determine whether we can reach any reliable
result.

One good candidate for our purpose is
\href{https://medium.freecodecamp.org/we-asked-20-000-people-who-they-are-and-how-theyre-learning-to-code-fff5d668969}{freeCodeCamp's
2017 New Coder Survey}.
\href{https://www.freecodecamp.org/}{freeCodeCamp} is a free e-learning
platform that offers courses on web development. Because they run
\href{https://medium.freecodecamp.org/}{a popular Medium publication}
(over 400,000 followers), their survey attracted new coders with varying
interests (not only web development), which is ideal for the purpose of
our analysis.

The survey data is publicly available in
\href{https://github.com/freeCodeCamp/2017-new-coder-survey}{this GitHub
repository}. Below, we'll do a quick exploration of the
\texttt{2017-fCC-New-Coders-Survey-Data.csv} file stored in the
\texttt{clean-data} folder of the repository we just mentioned. We'll
read in the file using the direct link
\href{https://raw.githubusercontent.com/freeCodeCamp/2017-new-coder-survey/master/clean-data/2017-fCC-New-Coders-Survey-Data.csv}{here}.

\begin{Shaded}
\begin{Highlighting}[]
\FunctionTok{library}\NormalTok{(readr)}

\NormalTok{fcc }\OtherTok{\textless{}{-}} \FunctionTok{read\_csv}\NormalTok{(}\StringTok{"2017{-}fCC{-}New{-}Coders{-}Survey{-}Data.csv"}\NormalTok{)}
\end{Highlighting}
\end{Shaded}

\begin{verbatim}
## Rows: 18175 Columns: 136
## -- Column specification --------------------------------------------------------
## Delimiter: ","
## chr   (27): BootcampName, CityPopulation, CodeEventOther, CommuteTime, Count...
## dbl  (105): Age, AttendedBootcamp, BootcampFinish, BootcampLoanYesNo, Bootca...
## dttm   (4): Part1EndTime, Part1StartTime, Part2EndTime, Part2StartTime
## 
## i Use `spec()` to retrieve the full column specification for this data.
## i Specify the column types or set `show_col_types = FALSE` to quiet this message.
\end{verbatim}

\begin{Shaded}
\begin{Highlighting}[]
\FunctionTok{dim}\NormalTok{(fcc)}
\end{Highlighting}
\end{Shaded}

\begin{verbatim}
## [1] 18175   136
\end{verbatim}

\begin{Shaded}
\begin{Highlighting}[]
\FunctionTok{head}\NormalTok{(fcc, }\DecValTok{5}\NormalTok{)}
\end{Highlighting}
\end{Shaded}

\begin{verbatim}
## # A tibble: 5 x 136
##     Age Attend~1 Bootc~2 Bootc~3 Bootc~4 Bootc~5 Child~6 CityP~7 CodeE~8 CodeE~9
##   <dbl>    <dbl>   <dbl>   <dbl> <chr>     <dbl>   <dbl> <chr>     <dbl>   <dbl>
## 1    27        0      NA      NA <NA>         NA      NA more t~      NA      NA
## 2    34        0      NA      NA <NA>         NA      NA less t~      NA      NA
## 3    21        0      NA      NA <NA>         NA      NA more t~      NA      NA
## 4    26        0      NA      NA <NA>         NA      NA betwee~      NA      NA
## 5    20        0      NA      NA <NA>         NA      NA betwee~      NA      NA
## # ... with 126 more variables: CodeEventFCC <dbl>, CodeEventGameJam <dbl>,
## #   CodeEventGirlDev <dbl>, CodeEventHackathons <dbl>, CodeEventMeetup <dbl>,
## #   CodeEventNodeSchool <dbl>, CodeEventNone <dbl>, CodeEventOther <chr>,
## #   CodeEventRailsBridge <dbl>, CodeEventRailsGirls <dbl>,
## #   CodeEventStartUpWknd <dbl>, CodeEventWkdBootcamps <dbl>,
## #   CodeEventWomenCode <dbl>, CodeEventWorkshops <dbl>, CommuteTime <chr>,
## #   CountryCitizen <chr>, CountryLive <chr>, EmploymentField <chr>, ...
\end{verbatim}

\hypertarget{checking-for-sample-representativity}{%
\subsection{Checking for Sample
Representativity}\label{checking-for-sample-representativity}}

As we mentioned in the introduction, most of our courses are on web and
mobile development, but we also cover many other domains, like data
science, game development, etc. For the purpose of our analysis, we want
to answer questions about a population of new coders that are interested
in the subjects we teach. We'd like to know:

\begin{itemize}
\tightlist
\item
  Where are these new coders located.
\item
  What locations have the greatest densities of new coders.
\item
  How much money they're willing to spend on learning.
\end{itemize}

So we first need to clarify whether the data set has the right
categories of people for our purpose. The \texttt{JobRoleInterest}
column describes for every participant the role(s) they'd be interested
in working in. If a participant is interested in working in a certain
domain, it means that they're also interested in learning about that
domain. So let's take a look at the frequency distribution table of this
column \footnote{We can use the
  \href{https://jofrhwld.github.io/teaching/courses/2017_lsa/lectures/Session_3.nb.html}{Split-and-Combine
  workflow}.} and determine whether the data we have is relevant.

\begin{Shaded}
\begin{Highlighting}[]
\CommentTok{\#split{-}and{-}combine workflow}
\FunctionTok{library}\NormalTok{(dplyr)}
\end{Highlighting}
\end{Shaded}

\begin{verbatim}
## 
## Attaching package: 'dplyr'
\end{verbatim}

\begin{verbatim}
## The following objects are masked from 'package:stats':
## 
##     filter, lag
\end{verbatim}

\begin{verbatim}
## The following objects are masked from 'package:base':
## 
##     intersect, setdiff, setequal, union
\end{verbatim}

\begin{Shaded}
\begin{Highlighting}[]
\NormalTok{fcc }\SpecialCharTok{\%\textgreater{}\%}
  \FunctionTok{group\_by}\NormalTok{(JobRoleInterest) }\SpecialCharTok{\%\textgreater{}\%}
  \FunctionTok{summarise}\NormalTok{(}\AttributeTok{freq =} \FunctionTok{n}\NormalTok{()}\SpecialCharTok{*}\DecValTok{100}\SpecialCharTok{/}\FunctionTok{nrow}\NormalTok{(fcc)) }\SpecialCharTok{\%\textgreater{}\%}
  \FunctionTok{arrange}\NormalTok{(}\FunctionTok{desc}\NormalTok{(freq))}
\end{Highlighting}
\end{Shaded}

\begin{verbatim}
## # A tibble: 3,212 x 2
##    JobRoleInterest                                       freq
##    <chr>                                                <dbl>
##  1 <NA>                                                61.5  
##  2 Full-Stack Web Developer                             4.53 
##  3 Front-End Web Developer                              2.48 
##  4 Data Scientist                                       0.836
##  5 Back-End Web Developer                               0.781
##  6 Mobile Developer                                     0.644
##  7 Game Developer                                       0.627
##  8 Information Security                                 0.506
##  9 Full-Stack Web Developer,   Front-End Web Developer  0.352
## 10 Front-End Web Developer, Full-Stack Web Developer    0.308
## # ... with 3,202 more rows
\end{verbatim}

The information in the table above is quite granular, but from a quick
scan it looks like:

\begin{itemize}
\tightlist
\item
  A lot of people are interested in web development (full-stack
  \emph{web development}, front-end \emph{web development} and back-end
  \emph{web development}).
\item
  A few people are interested in mobile development.
\item
  A few people are interested in domains other than web and mobile
  development.
\end{itemize}

It's also interesting to note that many respondents are interested in
more than one subject. It'd be useful to get a better picture of how
many people are interested in a single subject and how many have mixed
interests. Consequently, in the next code block, we'll:

\begin{itemize}
\tightlist
\item
  Split each string in the \texttt{JobRoleInterest} column to find the
  number of options for each participant.

  \begin{itemize}
  \tightlist
  \item
    We'll first drop the NA values \footnote{We can use the
      \href{https://tidyr.tidyverse.org/reference/drop_na.html}{\texttt{drop\_na()}
      function}.} because we cannot split NA values.
  \end{itemize}
\item
  Generate a frequency table for the variable describing the number of
  options \footnote{We can use the
    \href{https://www.rdocumentation.org/packages/stringr/versions/1.4.1/topics/str_split}{\texttt{stringr::str\_split()}
    function}.}.
\end{itemize}

\begin{Shaded}
\begin{Highlighting}[]
\CommentTok{\# Split each string in the \textquotesingle{}JobRoleInterest\textquotesingle{} column}
\NormalTok{splitted\_interests }\OtherTok{\textless{}{-}}\NormalTok{ fcc }\SpecialCharTok{\%\textgreater{}\%}
  \FunctionTok{select}\NormalTok{(JobRoleInterest) }\SpecialCharTok{\%\textgreater{}\%}
\NormalTok{  tidyr}\SpecialCharTok{::}\FunctionTok{drop\_na}\NormalTok{() }\SpecialCharTok{\%\textgreater{}\%}
  
\CommentTok{\#Tidyverse actually makes by default operation over columns, rowwise changes this behavior.}
\NormalTok{  rowwise }\SpecialCharTok{\%\textgreater{}\%}
  \FunctionTok{mutate}\NormalTok{(}\AttributeTok{opts =} \FunctionTok{length}\NormalTok{(stringr}\SpecialCharTok{::}\FunctionTok{str\_split}\NormalTok{(JobRoleInterest, }\StringTok{","}\NormalTok{)[[}\DecValTok{1}\NormalTok{]]))}
 \CommentTok{\# alternative implementation}
 \CommentTok{\#  mutate(opts = unlist( map(JobRoleInterest, function(x) length(str\_split(x,\textquotesingle{},\textquotesingle{})[[1]]))))}
 \CommentTok{\# then ungroup() is not needed later.}
  
\CommentTok{\# Frequency table for the var describing the number of options}
\NormalTok{n\_of\_options }\OtherTok{\textless{}{-}}\NormalTok{ splitted\_interests }\SpecialCharTok{\%\textgreater{}\%}
  \FunctionTok{ungroup}\NormalTok{() }\SpecialCharTok{\%\textgreater{}\%}  \CommentTok{\#this is needeed because we used the rowwise() function before}
  \FunctionTok{group\_by}\NormalTok{(opts) }\SpecialCharTok{\%\textgreater{}\%}
  \FunctionTok{summarize}\NormalTok{(}\AttributeTok{freq =} \FunctionTok{n}\NormalTok{()}\SpecialCharTok{*}\DecValTok{100}\SpecialCharTok{/}\FunctionTok{nrow}\NormalTok{(splitted\_interests))}

\NormalTok{n\_of\_options}
\end{Highlighting}
\end{Shaded}

\begin{verbatim}
## # A tibble: 13 x 2
##     opts    freq
##    <int>   <dbl>
##  1     1 31.7   
##  2     2 10.9   
##  3     3 15.9   
##  4     4 15.2   
##  5     5 12.0   
##  6     6  6.72  
##  7     7  3.86  
##  8     8  1.76  
##  9     9  0.987 
## 10    10  0.472 
## 11    11  0.186 
## 12    12  0.300 
## 13    13  0.0286
\end{verbatim}

It turns out that only 31.65\% of the participants have a clear idea
about what programming niche they'd like to work in, while the vast
majority of students have mixed interests. But given that we offer
courses on various subjects, the fact that new coders have mixed
interest might be actually good for us.

The focus of our courses is on web and mobile development, so let's find
out how many respondents chose at least one of these two options.

\begin{Shaded}
\begin{Highlighting}[]
\CommentTok{\# Frequency table (we can also use split{-}and{-}combine) str() will sum the result.}
\NormalTok{web\_or\_mobile }\OtherTok{\textless{}{-}}\NormalTok{ stringr}\SpecialCharTok{::}\FunctionTok{str\_detect}\NormalTok{(fcc}\SpecialCharTok{$}\NormalTok{JobRoleInterest, }\StringTok{"Web Developer|Mobile Developer"}\NormalTok{) }
\NormalTok{freq\_table }\OtherTok{\textless{}{-}} \FunctionTok{table}\NormalTok{(web\_or\_mobile)}
\NormalTok{freq\_table }\OtherTok{\textless{}{-}}\NormalTok{ freq\_table }\SpecialCharTok{*} \DecValTok{100} \SpecialCharTok{/} \FunctionTok{sum}\NormalTok{(freq\_table)}
\NormalTok{freq\_table}
\end{Highlighting}
\end{Shaded}

\begin{verbatim}
## web_or_mobile
##    FALSE     TRUE 
## 13.75858 86.24142
\end{verbatim}

\begin{Shaded}
\begin{Highlighting}[]
\CommentTok{\# Graph for the frequency table above}
\NormalTok{df }\OtherTok{\textless{}{-}}\NormalTok{ tibble}\SpecialCharTok{::}\FunctionTok{tibble}\NormalTok{(}\AttributeTok{x =} \FunctionTok{c}\NormalTok{(}\StringTok{"Other Subject"}\NormalTok{,}\StringTok{"Web or Mobile Developpement"}\NormalTok{),}
                       \AttributeTok{y =}\NormalTok{ freq\_table)}
 \FunctionTok{library}\NormalTok{(ggplot2)}
 \FunctionTok{ggplot}\NormalTok{(}\AttributeTok{data =}\NormalTok{ df, }\FunctionTok{aes}\NormalTok{(}\AttributeTok{x =}\NormalTok{ x, }\AttributeTok{y =}\NormalTok{ y, }\AttributeTok{fill =}\NormalTok{ x)) }\SpecialCharTok{+}
   \FunctionTok{geom\_histogram}\NormalTok{(}\AttributeTok{stat =} \StringTok{"identity"}\NormalTok{)}
\end{Highlighting}
\end{Shaded}

\begin{verbatim}
## Warning: Ignoring unknown parameters: binwidth, bins, pad
\end{verbatim}

\begin{verbatim}
## Don't know how to automatically pick scale for object of type table. Defaulting to continuous.
\end{verbatim}

\includegraphics{BestMarket_files/figure-latex/unnamed-chunk-4-1.pdf}

It turns out that most people in this survey (roughly 86\%) are
interested in either web or mobile development. These figures offer us a
strong reason to consider this sample representative for our population
of interest. We want to advertise our courses to people interested in
all sorts of programming niches but mostly web and mobile development.

Now we need to figure out what are the best markets to invest money in
for advertising our courses. We'd like to know:

\begin{itemize}
\tightlist
\item
  Where are these new coders located.
\item
  What are the locations with the greatest number of new coders.
\item
  How much money new coders are willing to spend on learning.
\end{itemize}

\hypertarget{new-coders---locations-and-densities}{%
\section{New Coders - Locations and
Densities}\label{new-coders---locations-and-densities}}

One indicator of a good market is the number of potential customers ---
the more potential customers in a market, the better. If our ads manage
to convince 10\% of the 5,000 potential customers in market A to buy our
product, then this is better than convincing 100\% of the 30 potential
customers in market B.

Let's begin with finding out where these new coders are located, and
what are the densities (how many new coders there are) for each
location. This should be a good start for finding out the best two
markets to run our ads campaign in.

The data set provides information about the location of each participant
at a country level. We can think of each country as an individual
market, so we can frame our goal as finding the two best countries to
advertise in.

We can start by examining the frequency distribution table of the
\texttt{CountryLive} variable, which describes what country each
participant lives in (not their origin country). We'll only consider
those participants who answered what role(s) they're interested in, to
make sure we work with a representative sample.

\begin{Shaded}
\begin{Highlighting}[]
\CommentTok{\# Isolate the participants that answered what role they\textquotesingle{}d be interested in}
\NormalTok{fcc\_good }\OtherTok{\textless{}{-}}\NormalTok{ fcc }\SpecialCharTok{\%\textgreater{}\%}
\NormalTok{  tidyr}\SpecialCharTok{::}\FunctionTok{drop\_na}\NormalTok{(JobRoleInterest) }
\CommentTok{\# Frequency tables with absolute and relative frequencies}
\CommentTok{\# Display the frequency tables in a more readable format}
\NormalTok{fcc\_good }\SpecialCharTok{\%\textgreater{}\%}
\FunctionTok{group\_by}\NormalTok{(CountryLive) }\SpecialCharTok{\%\textgreater{}\%}
\FunctionTok{summarise}\NormalTok{(}\StringTok{\textasciigrave{}}\AttributeTok{Absolute frequency}\StringTok{\textasciigrave{}} \OtherTok{=} \FunctionTok{n}\NormalTok{(),}
          \StringTok{\textasciigrave{}}\AttributeTok{Percentage}\StringTok{\textasciigrave{}} \OtherTok{=} \FunctionTok{n}\NormalTok{() }\SpecialCharTok{*} \DecValTok{100} \SpecialCharTok{/}  \FunctionTok{nrow}\NormalTok{(fcc\_good) ) }\SpecialCharTok{\%\textgreater{}\%}
  \FunctionTok{arrange}\NormalTok{(}\FunctionTok{desc}\NormalTok{(Percentage))}
\end{Highlighting}
\end{Shaded}

\begin{verbatim}
## # A tibble: 138 x 3
##    CountryLive              `Absolute frequency` Percentage
##    <chr>                                   <int>      <dbl>
##  1 United States of America                 3125      44.7 
##  2 India                                     528       7.55
##  3 United Kingdom                            315       4.51
##  4 Canada                                    260       3.72
##  5 <NA>                                      154       2.20
##  6 Poland                                    131       1.87
##  7 Brazil                                    129       1.84
##  8 Germany                                   125       1.79
##  9 Australia                                 112       1.60
## 10 Russia                                    102       1.46
## # ... with 128 more rows
\end{verbatim}

4.69\% of our potential customers are located in the US, and this
definitely seems like the most interesting market. India has the second
customer density, but it's just 7.55\%, which is not too far from the
United Kingdom (4.50\%) or Canada (3.71\%).

This is useful information, but we need to go more in depth than this
and figure out how much money people are actually willing to spend on
learning. Advertising in high-density markets where most people are only
willing to learn for free is extremely unlikely to be profitable for us.

\hypertarget{spending-money-for-learning}{%
\section{Spending Money for
Learning}\label{spending-money-for-learning}}

The \texttt{MoneyForLearning} column describes in American dollars the
amount of money spent by participants from the moment they started
coding until the moment they completed the survey. Our company sells
subscriptions at a price of \$59 per month, and for this reason we're
interested in finding out how much money each student spends per month.

We'll narrow down our analysis to only four countries: the US, India,
the United Kingdom, and Canada. We do this for two reasons:

\begin{itemize}
\tightlist
\item
  These are the countries having the highest frequency in the frequency
  table above, which means we have a decent amount of data for each.
\item
  Our courses are written in English, and English is an official
  language in all these four countries. The more people know English,
  the better our chances to target the right people with our ads.
\end{itemize}

Let's start with creating a new column that describes the amount of
money a student has spent per month so far. To do that, we'll need to
divide the \texttt{MoneyForLearning} column to the
\texttt{MonthsProgramming} column. The problem is that some students
answered that they have been learning to code for 0 months (it might be
that they have just started). To avoid dividing by 0, we'll replace 0
with 1 in the \texttt{MonthsProgramming} column.

\begin{Shaded}
\begin{Highlighting}[]
\CommentTok{\# Replace 0s with 1s to avoid division by 0}
\NormalTok{fcc\_good }\OtherTok{\textless{}{-}}\NormalTok{ fcc\_good }\SpecialCharTok{\%\textgreater{}\%}
  \FunctionTok{mutate}\NormalTok{(}\AttributeTok{MonthsProgramming =} \FunctionTok{replace}\NormalTok{(MonthsProgramming,  MonthsProgramming }\SpecialCharTok{==} \DecValTok{0}\NormalTok{, }\DecValTok{1}\NormalTok{) )}
\CommentTok{\# New column for the amount of money each student spends each month}
\NormalTok{fcc\_good }\OtherTok{\textless{}{-}}\NormalTok{ fcc\_good }\SpecialCharTok{\%\textgreater{}\%}
  \FunctionTok{mutate}\NormalTok{(}\AttributeTok{money\_per\_month =}\NormalTok{ MoneyForLearning}\SpecialCharTok{/}\NormalTok{MonthsProgramming) }
\NormalTok{fcc\_good }\SpecialCharTok{\%\textgreater{}\%}
  \FunctionTok{summarise}\NormalTok{(}\AttributeTok{na\_count =} \FunctionTok{sum}\NormalTok{(}\FunctionTok{is.na}\NormalTok{(money\_per\_month)) ) }\SpecialCharTok{\%\textgreater{}\%}
  \FunctionTok{pull}\NormalTok{(na\_count)}
\end{Highlighting}
\end{Shaded}

\begin{verbatim}
## [1] 675
\end{verbatim}

Let's keep only the rows that don't have NA values for the
\texttt{money\_per\_month} column.

\begin{Shaded}
\begin{Highlighting}[]
\CommentTok{\# Keep only the rows with non{-}NAs in the \textasciigrave{}money\_per\_month\textasciigrave{} column }
\NormalTok{fcc\_good  }\OtherTok{\textless{}{-}}\NormalTok{  fcc\_good }\SpecialCharTok{\%\textgreater{}\%}\NormalTok{ tidyr}\SpecialCharTok{::}\FunctionTok{drop\_na}\NormalTok{(money\_per\_month)}
\end{Highlighting}
\end{Shaded}

We want to group the data by country, and then measure the average
amount of money that students spend per month in each country. First,
let's remove the rows having \texttt{NA} values for the
\texttt{CountryLive} column, and check out if we still have enough data
for the four countries that interest us.

\begin{Shaded}
\begin{Highlighting}[]
\CommentTok{\# Remove the rows with NA values in \textquotesingle{}CountryLive\textquotesingle{}}
\NormalTok{fcc\_good  }\OtherTok{\textless{}{-}}\NormalTok{  fcc\_good }\SpecialCharTok{\%\textgreater{}\%}\NormalTok{ tidyr}\SpecialCharTok{::}\FunctionTok{drop\_na}\NormalTok{(CountryLive)}
\CommentTok{\# Frequency table to check if we still have enough data}
\NormalTok{fcc\_good }\SpecialCharTok{\%\textgreater{}\%} \FunctionTok{group\_by}\NormalTok{(CountryLive) }\SpecialCharTok{\%\textgreater{}\%}
  \FunctionTok{summarise}\NormalTok{(}\AttributeTok{freq =} \FunctionTok{n}\NormalTok{() ) }\SpecialCharTok{\%\textgreater{}\%}
  \FunctionTok{arrange}\NormalTok{(}\FunctionTok{desc}\NormalTok{(freq)) }\SpecialCharTok{\%\textgreater{}\%}
  \FunctionTok{head}\NormalTok{()}
\end{Highlighting}
\end{Shaded}

\begin{verbatim}
## # A tibble: 6 x 2
##   CountryLive               freq
##   <chr>                    <int>
## 1 United States of America  2933
## 2 India                      463
## 3 United Kingdom             279
## 4 Canada                     240
## 5 Poland                     122
## 6 Germany                    114
\end{verbatim}

This should be enough, so let's compute the average value spent per
month in each country by a student. We'll compute the average using the
mean.

\begin{Shaded}
\begin{Highlighting}[]
\CommentTok{\# Mean sum of money spent by students each month}
\NormalTok{countries\_mean  }\OtherTok{\textless{}{-}}\NormalTok{  fcc\_good }\SpecialCharTok{\%\textgreater{}\%} 
  \FunctionTok{filter}\NormalTok{(CountryLive }\SpecialCharTok{==} \StringTok{\textquotesingle{}United States of America\textquotesingle{}} \SpecialCharTok{|}\NormalTok{ CountryLive }\SpecialCharTok{==} \StringTok{\textquotesingle{}India\textquotesingle{}} \SpecialCharTok{|}\NormalTok{ CountryLive }\SpecialCharTok{==} \StringTok{\textquotesingle{}United Kingdom\textquotesingle{}}\SpecialCharTok{|}\NormalTok{CountryLive }\SpecialCharTok{==} \StringTok{\textquotesingle{}Canada\textquotesingle{}}\NormalTok{) }\SpecialCharTok{\%\textgreater{}\%}
  \FunctionTok{group\_by}\NormalTok{(CountryLive) }\SpecialCharTok{\%\textgreater{}\%}
  \FunctionTok{summarize}\NormalTok{(}\AttributeTok{mean =} \FunctionTok{mean}\NormalTok{(money\_per\_month)) }\SpecialCharTok{\%\textgreater{}\%}
  \FunctionTok{arrange}\NormalTok{(}\FunctionTok{desc}\NormalTok{(mean))}
\NormalTok{countries\_mean}
\end{Highlighting}
\end{Shaded}

\begin{verbatim}
## # A tibble: 4 x 2
##   CountryLive               mean
##   <chr>                    <dbl>
## 1 United States of America 228. 
## 2 India                    135. 
## 3 Canada                   114. 
## 4 United Kingdom            45.5
\end{verbatim}

The results for the United Kingdom and Canada are a bit surprising
relative to the values we see for India. If we considered a few
socio-economical metrics (like \href{https://bit.ly/2I3cukh}{GDP per
capita}), we'd intuitively expect people in the UK and Canada to spend
more on learning than people in India.

It might be that we don't have have enough representative data for the
United Kingdom and Canada, or we have some outliers (maybe coming from
wrong survey answers) making the mean too large for India, or too low
for the UK and Canada. Or it might be that the results are correct.

\hypertarget{dealing-with-extreme-outliers}{%
\subsection{Dealing with Extreme
Outliers}\label{dealing-with-extreme-outliers}}

Let's use box plots to visualize the distribution of the
\texttt{money\_per\_month} variable for each country.

\begin{Shaded}
\begin{Highlighting}[]
\CommentTok{\# Isolate only the countries of interest}
\NormalTok{only\_4  }\OtherTok{\textless{}{-}}\NormalTok{  fcc\_good }\SpecialCharTok{\%\textgreater{}\%} 
  \FunctionTok{filter}\NormalTok{(CountryLive }\SpecialCharTok{==} \StringTok{\textquotesingle{}United States of America\textquotesingle{}} \SpecialCharTok{|}\NormalTok{ CountryLive }\SpecialCharTok{==} \StringTok{\textquotesingle{}India\textquotesingle{}} \SpecialCharTok{|}\NormalTok{ CountryLive }\SpecialCharTok{==} \StringTok{\textquotesingle{}United Kingdom\textquotesingle{}}\SpecialCharTok{|}\NormalTok{CountryLive }\SpecialCharTok{==} \StringTok{\textquotesingle{}Canada\textquotesingle{}}\NormalTok{)}
\CommentTok{\# Since maybe, we will remove elements from the database, }
\CommentTok{\# we add an index column containing the number of each row. }
\CommentTok{\# Hence, we will have a match with the original database in case of some indexes.}
\NormalTok{only\_4 }\OtherTok{\textless{}{-}}\NormalTok{ only\_4 }\SpecialCharTok{\%\textgreater{}\%}
  \FunctionTok{mutate}\NormalTok{(}\AttributeTok{index =} \FunctionTok{row\_number}\NormalTok{())}
\CommentTok{\# Box plots to visualize distributions}
\FunctionTok{ggplot}\NormalTok{( }\AttributeTok{data =}\NormalTok{ only\_4, }\FunctionTok{aes}\NormalTok{(}\AttributeTok{x =}\NormalTok{ CountryLive, }\AttributeTok{y =}\NormalTok{ money\_per\_month)) }\SpecialCharTok{+}
  \FunctionTok{geom\_boxplot}\NormalTok{() }\SpecialCharTok{+}
  \FunctionTok{ggtitle}\NormalTok{(}\StringTok{"Money Spent Per Month Per Country}\SpecialCharTok{\textbackslash{}n}\StringTok{(Distributions)"}\NormalTok{) }\SpecialCharTok{+}
  \FunctionTok{xlab}\NormalTok{(}\StringTok{"Country"}\NormalTok{) }\SpecialCharTok{+}
  \FunctionTok{ylab}\NormalTok{(}\StringTok{"Money per month (US dollars)"}\NormalTok{) }\SpecialCharTok{+}
  \FunctionTok{theme\_bw}\NormalTok{()}
\end{Highlighting}
\end{Shaded}

\includegraphics{BestMarket_files/figure-latex/unnamed-chunk-10-1.pdf}

It's hard to see on the plot above if there's anything wrong with the
data for the United Kingdom, India, or Canada, but we can see
immediately that there's something really off for the US: two persons
spend each month \$50,000 or more for learning. This is not impossible,
but it seems extremely unlikely, so we'll remove every value that goes
over \$20,000 per month.

\begin{Shaded}
\begin{Highlighting}[]
\CommentTok{\# Isolate only those participants who spend less than 10,000 per month}
\NormalTok{fcc\_good  }\OtherTok{\textless{}{-}}\NormalTok{ fcc\_good }\SpecialCharTok{\%\textgreater{}\%} 
  \FunctionTok{filter}\NormalTok{(money\_per\_month }\SpecialCharTok{\textless{}} \DecValTok{20000}\NormalTok{)}
\end{Highlighting}
\end{Shaded}

Now let's recompute the mean values and plot the box plots again.

\begin{Shaded}
\begin{Highlighting}[]
\CommentTok{\# Mean sum of money spent by students each month}
\NormalTok{countries\_mean }\OtherTok{=}\NormalTok{ fcc\_good }\SpecialCharTok{\%\textgreater{}\%} 
  \FunctionTok{filter}\NormalTok{(CountryLive }\SpecialCharTok{==} \StringTok{\textquotesingle{}United States of America\textquotesingle{}} \SpecialCharTok{|}\NormalTok{ CountryLive }\SpecialCharTok{==} \StringTok{\textquotesingle{}India\textquotesingle{}} \SpecialCharTok{|}\NormalTok{ CountryLive }\SpecialCharTok{==} \StringTok{\textquotesingle{}United Kingdom\textquotesingle{}}\SpecialCharTok{|}\NormalTok{CountryLive }\SpecialCharTok{==} \StringTok{\textquotesingle{}Canada\textquotesingle{}}\NormalTok{) }\SpecialCharTok{\%\textgreater{}\%}
  \FunctionTok{group\_by}\NormalTok{(CountryLive) }\SpecialCharTok{\%\textgreater{}\%}
  \FunctionTok{summarize}\NormalTok{(}\AttributeTok{mean =} \FunctionTok{mean}\NormalTok{(money\_per\_month)) }\SpecialCharTok{\%\textgreater{}\%}
  \FunctionTok{arrange}\NormalTok{(}\FunctionTok{desc}\NormalTok{(mean))}
\NormalTok{countries\_mean}
\end{Highlighting}
\end{Shaded}

\begin{verbatim}
## # A tibble: 4 x 2
##   CountryLive               mean
##   <chr>                    <dbl>
## 1 United States of America 184. 
## 2 India                    135. 
## 3 Canada                   114. 
## 4 United Kingdom            45.5
\end{verbatim}

\begin{Shaded}
\begin{Highlighting}[]
\CommentTok{\# Isolate only the countries of interest}
\NormalTok{only\_4  }\OtherTok{\textless{}{-}}\NormalTok{  fcc\_good }\SpecialCharTok{\%\textgreater{}\%} 
  \FunctionTok{filter}\NormalTok{(CountryLive }\SpecialCharTok{==} \StringTok{\textquotesingle{}United States of America\textquotesingle{}} \SpecialCharTok{|}\NormalTok{ CountryLive }\SpecialCharTok{==} \StringTok{\textquotesingle{}India\textquotesingle{}} \SpecialCharTok{|}\NormalTok{ CountryLive }\SpecialCharTok{==} \StringTok{\textquotesingle{}United Kingdom\textquotesingle{}}\SpecialCharTok{|}\NormalTok{CountryLive }\SpecialCharTok{==} \StringTok{\textquotesingle{}Canada\textquotesingle{}}\NormalTok{) }\SpecialCharTok{\%\textgreater{}\%}
  \FunctionTok{mutate}\NormalTok{(}\AttributeTok{index =} \FunctionTok{row\_number}\NormalTok{())}
\CommentTok{\# Box plots to visualize distributions}
\FunctionTok{ggplot}\NormalTok{( }\AttributeTok{data =}\NormalTok{ only\_4, }\FunctionTok{aes}\NormalTok{(}\AttributeTok{x =}\NormalTok{ CountryLive, }\AttributeTok{y =}\NormalTok{ money\_per\_month)) }\SpecialCharTok{+}
  \FunctionTok{geom\_boxplot}\NormalTok{() }\SpecialCharTok{+}
  \FunctionTok{ggtitle}\NormalTok{(}\StringTok{"Money Spent Per Month Per Country}\SpecialCharTok{\textbackslash{}n}\StringTok{(Distributions)"}\NormalTok{) }\SpecialCharTok{+}
  \FunctionTok{xlab}\NormalTok{(}\StringTok{"Country"}\NormalTok{) }\SpecialCharTok{+}
  \FunctionTok{ylab}\NormalTok{(}\StringTok{"Money per month (US dollars)"}\NormalTok{) }\SpecialCharTok{+}
  \FunctionTok{theme\_bw}\NormalTok{()}
\end{Highlighting}
\end{Shaded}

\includegraphics{BestMarket_files/figure-latex/unnamed-chunk-13-1.pdf}

We can see a few extreme outliers for India (values over \$2,500 per
month), but it's unclear whether this is good data or not. Maybe these
persons attended several bootcamps, which tend to be very expensive.
Let's examine these two data points to see if we can find anything
relevant.

\begin{Shaded}
\begin{Highlighting}[]
\CommentTok{\# Inspect the extreme outliers for India}
\NormalTok{india\_outliers  }\OtherTok{\textless{}{-}}\NormalTok{  only\_4 }\SpecialCharTok{\%\textgreater{}\%}
  \FunctionTok{filter}\NormalTok{(CountryLive }\SpecialCharTok{==} \StringTok{\textquotesingle{}India\textquotesingle{}} \SpecialCharTok{\&} 
\NormalTok{           money\_per\_month }\SpecialCharTok{\textgreater{}=} \DecValTok{2500}\NormalTok{)}
\NormalTok{india\_outliers}
\end{Highlighting}
\end{Shaded}

\begin{verbatim}
## # A tibble: 6 x 138
##     Age Attend~1 Bootc~2 Bootc~3 Bootc~4 Bootc~5 Child~6 CityP~7 CodeE~8 CodeE~9
##   <dbl>    <dbl>   <dbl>   <dbl> <chr>     <dbl>   <dbl> <chr>     <dbl>   <dbl>
## 1    24        0      NA      NA <NA>         NA      NA betwee~      NA      NA
## 2    20        0      NA      NA <NA>         NA      NA more t~      NA      NA
## 3    28        0      NA      NA <NA>         NA      NA betwee~       1      NA
## 4    22        0      NA      NA <NA>         NA      NA more t~      NA      NA
## 5    19        0      NA      NA <NA>         NA      NA more t~      NA      NA
## 6    27        0      NA      NA <NA>         NA      NA more t~      NA      NA
## # ... with 128 more variables: CodeEventFCC <dbl>, CodeEventGameJam <dbl>,
## #   CodeEventGirlDev <dbl>, CodeEventHackathons <dbl>, CodeEventMeetup <dbl>,
## #   CodeEventNodeSchool <dbl>, CodeEventNone <dbl>, CodeEventOther <chr>,
## #   CodeEventRailsBridge <dbl>, CodeEventRailsGirls <dbl>,
## #   CodeEventStartUpWknd <dbl>, CodeEventWkdBootcamps <dbl>,
## #   CodeEventWomenCode <dbl>, CodeEventWorkshops <dbl>, CommuteTime <chr>,
## #   CountryCitizen <chr>, CountryLive <chr>, EmploymentField <chr>, ...
\end{verbatim}

It seems that neither participant attended a bootcamp. Overall, it's
really hard to figure out from the data whether these persons really
spent that much money with learning. The actual question of the survey
was \emph{``Aside from university tuition, about how much money have you
spent on learning to code so far (in US dollars)?''}, so they might have
misunderstood and thought university tuition is included. It seems safer
to remove these six rows.

\begin{Shaded}
\begin{Highlighting}[]
\CommentTok{\# Remove the outliers for India}
\NormalTok{only\_4 }\OtherTok{\textless{}{-}}\NormalTok{  only\_4 }\SpecialCharTok{\%\textgreater{}\%} 
  \FunctionTok{filter}\NormalTok{(}\SpecialCharTok{!}\NormalTok{(index }\SpecialCharTok{\%in\%}\NormalTok{ india\_outliers}\SpecialCharTok{$}\NormalTok{index))}
\end{Highlighting}
\end{Shaded}

Looking back at the box plot above, we can also see more extreme
outliers for the US (values over \$6,000 per month). Let's examine these
participants in more detail.

\begin{Shaded}
\begin{Highlighting}[]
\CommentTok{\# Examine the extreme outliers for the US}
\NormalTok{us\_outliers }\OtherTok{=}\NormalTok{ only\_4 }\SpecialCharTok{\%\textgreater{}\%}
  \FunctionTok{filter}\NormalTok{(CountryLive }\SpecialCharTok{==} \StringTok{\textquotesingle{}United States of America\textquotesingle{}} \SpecialCharTok{\&} 
\NormalTok{           money\_per\_month }\SpecialCharTok{\textgreater{}=} \DecValTok{6000}\NormalTok{)}
\NormalTok{us\_outliers}
\end{Highlighting}
\end{Shaded}

\begin{verbatim}
## # A tibble: 11 x 138
##      Age Atten~1 Bootc~2 Bootc~3 Bootc~4 Bootc~5 Child~6 CityP~7 CodeE~8 CodeE~9
##    <dbl>   <dbl>   <dbl>   <dbl> <chr>     <dbl>   <dbl> <chr>     <dbl>   <dbl>
##  1    26       1       0       0 The Co~       1      NA more t~       1      NA
##  2    32       1       0       0 The Ir~       1      NA betwee~      NA      NA
##  3    34       1       1       0 We Can~       1      NA more t~      NA      NA
##  4    31       0      NA      NA <NA>         NA      NA betwee~      NA      NA
##  5    46       1       1       1 Sabio.~       0      NA betwee~      NA      NA
##  6    32       0      NA      NA <NA>         NA      NA more t~       1      NA
##  7    26       1       0       1 Codeup        0      NA more t~      NA      NA
##  8    33       1       0       1 Grand ~       1      NA betwee~      NA      NA
##  9    29       0      NA      NA <NA>         NA       2 more t~      NA      NA
## 10    27       0      NA      NA <NA>         NA       1 more t~      NA      NA
## 11    50       0      NA      NA <NA>         NA       2 less t~      NA      NA
## # ... with 128 more variables: CodeEventFCC <dbl>, CodeEventGameJam <dbl>,
## #   CodeEventGirlDev <dbl>, CodeEventHackathons <dbl>, CodeEventMeetup <dbl>,
## #   CodeEventNodeSchool <dbl>, CodeEventNone <dbl>, CodeEventOther <chr>,
## #   CodeEventRailsBridge <dbl>, CodeEventRailsGirls <dbl>,
## #   CodeEventStartUpWknd <dbl>, CodeEventWkdBootcamps <dbl>,
## #   CodeEventWomenCode <dbl>, CodeEventWorkshops <dbl>, CommuteTime <chr>,
## #   CountryCitizen <chr>, CountryLive <chr>, EmploymentField <chr>, ...
\end{verbatim}

\begin{Shaded}
\begin{Highlighting}[]
\NormalTok{only\_4  }\OtherTok{\textless{}{-}}\NormalTok{  only\_4 }\SpecialCharTok{\%\textgreater{}\%} 
  \FunctionTok{filter}\NormalTok{(}\SpecialCharTok{!}\NormalTok{(index }\SpecialCharTok{\%in\%}\NormalTok{ us\_outliers}\SpecialCharTok{$}\NormalTok{index))}
\end{Highlighting}
\end{Shaded}

Out of these 11 extreme outliers, six people attended bootcamps, which
justify the large sums of money spent on learning. For the other five,
it's hard to figure out from the data where they could have spent that
much money on learning. Consequently, we'll remove those rows where
participants reported thaT they spend \$6,000 each month, but they have
never attended a bootcamp.

Also, the data shows that eight respondents had been programming for no
more than three months when they completed the survey. They most likely
paid a large sum of money for a bootcamp that was going to last for
several months, so the amount of money spent per month is unrealistic
and should be significantly lower (because they probably didn't spend
anything for the next couple of months after the survey). As a
consequence, we'll remove every these eight outliers.

In the next code block, we'll remove respondents that:

\begin{itemize}
\tightlist
\item
  Didn't attend bootcamps.
\item
  Had been programming for three months or less when at the time they
  completed the survey.
\end{itemize}

\begin{Shaded}
\begin{Highlighting}[]
\CommentTok{\# Remove the respondents who didn\textquotesingle{}t attenD a bootcamp}
\NormalTok{no\_bootcamp }\OtherTok{\textless{}{-}}\NormalTok{ only\_4 }\SpecialCharTok{\%\textgreater{}\%}
    \FunctionTok{filter}\NormalTok{(CountryLive }\SpecialCharTok{==} \StringTok{\textquotesingle{}United States of America\textquotesingle{}} \SpecialCharTok{\&} 
\NormalTok{           money\_per\_month }\SpecialCharTok{\textgreater{}=} \DecValTok{6000} \SpecialCharTok{\&}
\NormalTok{             AttendedBootcamp }\SpecialCharTok{==} \DecValTok{0}\NormalTok{)}
\NormalTok{only\_4\_  }\OtherTok{\textless{}{-}}\NormalTok{  only\_4 }\SpecialCharTok{\%\textgreater{}\%} 
  \FunctionTok{filter}\NormalTok{(}\SpecialCharTok{!}\NormalTok{(index }\SpecialCharTok{\%in\%}\NormalTok{ no\_bootcamp}\SpecialCharTok{$}\NormalTok{index))}
\CommentTok{\# Remove the respondents that had been programming for less than 3 months}
\NormalTok{less\_than\_3\_months }\OtherTok{\textless{}{-}}\NormalTok{ only\_4 }\SpecialCharTok{\%\textgreater{}\%}
    \FunctionTok{filter}\NormalTok{(CountryLive }\SpecialCharTok{==} \StringTok{\textquotesingle{}United States of America\textquotesingle{}} \SpecialCharTok{\&} 
\NormalTok{           money\_per\_month }\SpecialCharTok{\textgreater{}=} \DecValTok{6000} \SpecialCharTok{\&}
\NormalTok{           MonthsProgramming }\SpecialCharTok{\textless{}=} \DecValTok{3}\NormalTok{)}
\NormalTok{only\_4  }\OtherTok{\textless{}{-}}\NormalTok{  only\_4 }\SpecialCharTok{\%\textgreater{}\%} 
  \FunctionTok{filter}\NormalTok{(}\SpecialCharTok{!}\NormalTok{(index }\SpecialCharTok{\%in\%}\NormalTok{ less\_than\_3\_months}\SpecialCharTok{$}\NormalTok{index))}
\end{Highlighting}
\end{Shaded}

Looking again at the last box plot above, we can also see an extreme
outlier for Canada --- a person who spends roughly \$5,000 per month.
Let's examine this person in more depth.

\begin{Shaded}
\begin{Highlighting}[]
\CommentTok{\# Examine the extreme outliers for Canada}
\NormalTok{canada\_outliers }\OtherTok{=}\NormalTok{ only\_4 }\SpecialCharTok{\%\textgreater{}\%}
  \FunctionTok{filter}\NormalTok{(CountryLive }\SpecialCharTok{==} \StringTok{\textquotesingle{}Canada\textquotesingle{}} \SpecialCharTok{\&} 
\NormalTok{           money\_per\_month }\SpecialCharTok{\textgreater{}=} \DecValTok{4500} \SpecialCharTok{\&}
\NormalTok{           MonthsProgramming }\SpecialCharTok{\textless{}=} \DecValTok{3}\NormalTok{)}
\NormalTok{canada\_outliers}
\end{Highlighting}
\end{Shaded}

\begin{verbatim}
## # A tibble: 1 x 138
##     Age Attend~1 Bootc~2 Bootc~3 Bootc~4 Bootc~5 Child~6 CityP~7 CodeE~8 CodeE~9
##   <dbl>    <dbl>   <dbl>   <dbl> <chr>     <dbl>   <dbl> <chr>     <dbl>   <dbl>
## 1    24        1       0       0 Bloc.io       1      NA more t~       1      NA
## # ... with 128 more variables: CodeEventFCC <dbl>, CodeEventGameJam <dbl>,
## #   CodeEventGirlDev <dbl>, CodeEventHackathons <dbl>, CodeEventMeetup <dbl>,
## #   CodeEventNodeSchool <dbl>, CodeEventNone <dbl>, CodeEventOther <chr>,
## #   CodeEventRailsBridge <dbl>, CodeEventRailsGirls <dbl>,
## #   CodeEventStartUpWknd <dbl>, CodeEventWkdBootcamps <dbl>,
## #   CodeEventWomenCode <dbl>, CodeEventWorkshops <dbl>, CommuteTime <chr>,
## #   CountryCitizen <chr>, CountryLive <chr>, EmploymentField <chr>, ...
\end{verbatim}

Here, the situation is similar to some of the US respondents --- this
participant had been programming for no more than two months when he
completed the survey. He seems to have paid a large sum of money in the
beginning to enroll in a bootcamp, and then he probably didn't spend
anything for the next couple of months after the survey. We'll take the
same approach here as for the US and remove this outlier.

\begin{Shaded}
\begin{Highlighting}[]
\CommentTok{\# Remove the extreme outliers for Canada}
\NormalTok{only\_4  }\OtherTok{\textless{}{-}}\NormalTok{  only\_4 }\SpecialCharTok{\%\textgreater{}\%} 
  \FunctionTok{filter}\NormalTok{(}\SpecialCharTok{!}\NormalTok{(index }\SpecialCharTok{\%in\%}\NormalTok{ canada\_outliers}\SpecialCharTok{$}\NormalTok{index))}
\end{Highlighting}
\end{Shaded}

Let's recompute the mean values and generate the final box plots.

\begin{Shaded}
\begin{Highlighting}[]
\CommentTok{\# Mean sum of money spent by students each month}
\NormalTok{countries\_mean }\OtherTok{\textless{}{-}}\NormalTok{ only\_4 }\SpecialCharTok{\%\textgreater{}\%}
  \FunctionTok{group\_by}\NormalTok{(CountryLive) }\SpecialCharTok{\%\textgreater{}\%}
  \FunctionTok{summarize}\NormalTok{(}\AttributeTok{mean =} \FunctionTok{mean}\NormalTok{(money\_per\_month)) }\SpecialCharTok{\%\textgreater{}\%}
  \FunctionTok{arrange}\NormalTok{(}\FunctionTok{desc}\NormalTok{(mean))}
\NormalTok{countries\_mean}
\end{Highlighting}
\end{Shaded}

\begin{verbatim}
## # A tibble: 4 x 2
##   CountryLive               mean
##   <chr>                    <dbl>
## 1 United States of America 143. 
## 2 Canada                    93.1
## 3 India                     65.8
## 4 United Kingdom            45.5
\end{verbatim}

\begin{Shaded}
\begin{Highlighting}[]
\CommentTok{\# Box plots to visualize distributions}
\FunctionTok{ggplot}\NormalTok{( }\AttributeTok{data =}\NormalTok{ only\_4, }\FunctionTok{aes}\NormalTok{(}\AttributeTok{x =}\NormalTok{ CountryLive, }\AttributeTok{y =}\NormalTok{ money\_per\_month)) }\SpecialCharTok{+}
  \FunctionTok{geom\_boxplot}\NormalTok{() }\SpecialCharTok{+}
  \FunctionTok{ggtitle}\NormalTok{(}\StringTok{"Money Spent Per Month Per Country}\SpecialCharTok{\textbackslash{}n}\StringTok{(Distributions)"}\NormalTok{) }\SpecialCharTok{+}
  \FunctionTok{xlab}\NormalTok{(}\StringTok{"Country"}\NormalTok{) }\SpecialCharTok{+}
  \FunctionTok{ylab}\NormalTok{(}\StringTok{"Money per month (US dollars)"}\NormalTok{) }\SpecialCharTok{+}
  \FunctionTok{theme\_bw}\NormalTok{()}
\end{Highlighting}
\end{Shaded}

\includegraphics{BestMarket_files/figure-latex/unnamed-chunk-21-1.pdf}
\# Choosing the Two Best Markets

Obviously, one country we should advertise in is the US. Lots of new
coders live there and they are willing to pay a good amount of money
each month (roughly \$143).

We sell subscriptions at a price of \$59 per month, and Canada seems to
be the best second choice because people there are willing to pay
roughly \$93 per month, compared to India (\$66) and the United Kingdom
(\$45).

The data suggests strongly that we shouldn't advertise in the UK, but
let's take a second look at India before deciding to choose Canada as
our second best choice:

\begin{itemize}
\tightlist
\item
  \$59 doesn't seem like an expensive sum for people in India since they
  spend on average \$66 each month.
\item
  We have almost twice as more potential customers in India than we have
  in Canada:
\end{itemize}

\begin{Shaded}
\begin{Highlighting}[]
\CommentTok{\# Frequency table for the \textquotesingle{}CountryLive\textquotesingle{} column}
\NormalTok{only\_4 }\SpecialCharTok{\%\textgreater{}\%} \FunctionTok{group\_by}\NormalTok{(CountryLive) }\SpecialCharTok{\%\textgreater{}\%}
  \FunctionTok{summarise}\NormalTok{(}\AttributeTok{freq =} \FunctionTok{n}\NormalTok{() }\SpecialCharTok{*} \DecValTok{100} \SpecialCharTok{/} \FunctionTok{nrow}\NormalTok{(only\_4) ) }\SpecialCharTok{\%\textgreater{}\%}
  \FunctionTok{arrange}\NormalTok{(}\FunctionTok{desc}\NormalTok{(freq)) }\SpecialCharTok{\%\textgreater{}\%}
  \FunctionTok{head}\NormalTok{()}
\end{Highlighting}
\end{Shaded}

\begin{verbatim}
## # A tibble: 4 x 2
##   CountryLive               freq
##   <chr>                    <dbl>
## 1 United States of America 75.0 
## 2 India                    11.7 
## 3 United Kingdom            7.16
## 4 Canada                    6.14
\end{verbatim}

\begin{Shaded}
\begin{Highlighting}[]
\CommentTok{\# Frequency table to check if we still have enough data}
\NormalTok{only\_4 }\SpecialCharTok{\%\textgreater{}\%} \FunctionTok{group\_by}\NormalTok{(CountryLive) }\SpecialCharTok{\%\textgreater{}\%}
  \FunctionTok{summarise}\NormalTok{(}\AttributeTok{freq =} \FunctionTok{n}\NormalTok{() ) }\SpecialCharTok{\%\textgreater{}\%}
  \FunctionTok{arrange}\NormalTok{(}\FunctionTok{desc}\NormalTok{(freq)) }\SpecialCharTok{\%\textgreater{}\%}
  \FunctionTok{head}\NormalTok{()}
\end{Highlighting}
\end{Shaded}

\begin{verbatim}
## # A tibble: 4 x 2
##   CountryLive               freq
##   <chr>                    <int>
## 1 United States of America  2920
## 2 India                      457
## 3 United Kingdom             279
## 4 Canada                     239
\end{verbatim}

So it's not crystal clear what to choose between Canada and India.
Although it seems more tempting to choose Canada, there are good chances
that India might actually be a better choice because of the large number
of potential customers.

At this point, it seems that we have several options:

\begin{enumerate}
\def\labelenumi{\arabic{enumi}.}
\tightlist
\item
  Advertise in the US, India, and Canada by splitting the advertisement
  budget in various combinations:

  \begin{itemize}
  \tightlist
  \item
    60\% for the US, 25\% for India, 15\% for Canada.
  \item
    50\% for the US, 30\% for India, 20\% for Canada; etc.
  \end{itemize}
\item
  Advertise only in the US and India, or the US and Canada. Again, it
  makes sense to split the advertisement budget unequally. For instance:

  \begin{itemize}
  \tightlist
  \item
    70\% for the US, and 30\% for India.
  \item
    65\% for the US, and 35\% for Canada; etc.
  \end{itemize}
\item
  Advertise only in the US.
\end{enumerate}

At this point, it's probably best to send our analysis to the marketing
team and let them use their domain knowledge to decide. They might want
to do some extra surveys in India and Canada and then get back to us for
analyzing the new survey data.

\hypertarget{conclusion}{%
\section{Conclusion}\label{conclusion}}

In this project, we analyzed survey data from new coders to find the
best two markets to advertise in. The only solid conclusion we reached
is that the US would be a good market to advertise in.

For the second best market, it wasn't clear-cut what to choose between
India and Canada. We decided to send the results to the marketing team
so they can use their domain knowledge to take the best decision.

\hypertarget{documentation}{%
\section{Documentation}\label{documentation}}

\end{document}
